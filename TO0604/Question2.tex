\documentclass{article}

\usepackage{amsmath, amsthm, amssymb, amsfonts}
\usepackage{thmtools}
\usepackage{graphicx}
\usepackage{setspace}
\usepackage{geometry}
\usepackage{float}
\usepackage{hyperref}
\usepackage[utf8]{inputenc}
\usepackage[english]{babel}
\usepackage{framed}
\usepackage[dvipsnames]{xcolor}
\usepackage{tcolorbox}

\colorlet{LightGray}{White!90!Periwinkle}
\colorlet{LightOrange}{Orange!15}
\colorlet{LightGreen}{Green!15}

\newcommand{\HRule}[1]{\rule{\linewidth}{#1}}

\declaretheoremstyle[name=Théorème,]{thmsty}
\declaretheorem[style=thmsty,numberwithin=section]{theorem}
\tcolorboxenvironment{theorem}{colback=LightGray}

\declaretheoremstyle[name=Proposition,]{prosty}
\declaretheorem[style=prosty,numberlike=theorem]{proposition}
\tcolorboxenvironment{proposition}{colback=LightOrange}

\declaretheoremstyle[name=Propriété,]{prcpsty}
\declaretheorem[style=prcpsty,numberlike=theorem]{principle}
\tcolorboxenvironment{principle}{colback=LightOrange}

\declaretheoremstyle[name=Définition,]{defisty}
\declaretheorem[style=defisty,numberlike=theorem]{definition}
\tcolorboxenvironment{definition}{colback=LightGreen}

\setstretch{1.2}
\geometry{
    textheight=9in,
    textwidth=5.5in,
    top=1in,
    headheight=12pt,
    headsep=25pt,
    footskip=30pt
}

% ------------------------------------------------------------------------------

\begin{document}

% ------------------------------------------------------------------------------
% Cover Page and ToC
% ------------------------------------------------------------------------------

\title{ \normalsize \textsc{}
        \\ [2.0cm]
        \HRule{1.5pt} \
        \LARGE \textbf{\uppercase{Topologie}
        \HRule{2.0pt} \\ [0.6cm] \LARGE{Question 2} \vspace*{10\baselineskip}}
        }
\date{}
\author{\textbf{Author} \ 
        Brandon Lin}
\maketitle
\newpage

\tableofcontents
\newpage

% ------------------------------------------------------------------------------

\section{Exercice 2}

\textit{Notations}

\begin{itemize}
    \item $(x_n^{(p)})_{n \in \mathbb{N} } \in K_p^{\mathbb{N} }$ une suite de $K_p$
    \item $(x_{\phi_p(n)}^{(p)})$ une sous-suite de $(x_n^{(p)})$
    \item $\lambda_p$ un vecteur dans $K_p$
    \item $(x_n)_{n \in \mathbb{N} } \in K^\mathbb{N}$ une suite de $K$
        \[
            x_n = \left( x_n^{(1)} , \ldots, x_n^{(p)}\right) \in \prod_{t=1}^p K_t^\mathbb{N} 
        \]
        
    \item $(x_{\phi(n)})$ une sous-suite de $(x_n)$
    \item $\lambda$ un vecteur dans $K$

\end{itemize}

\noindent \textit{Résolution générale}

Comme $(E_0, d_0),\ldots,(E_n,d_n)$ sont des espaces métriques compactes, soit $(x_n)_{n \in \mathbb{N} } \in K^\mathbb{N}$ une suite de $K$
        \[
            x_n = \left( x_n^{(0)} , \ldots, x_n^{(p)}\right) \in \prod_{t=0}^p K_t^\mathbb{N} 
        \]

De la suite $\left( x_n^{(0)} \right) \in K_0^\mathbb{N}$ on extrait une sous-suite $\left( x_{\phi_0(n)}^{(0)} \right)$ qui tend vers $\lambda_0 \in K_0$.

De la suite $(x_{\phi_0(n)}^{(0)})$, on extrait une sous-suite $\left( x_{\phi_0 \circ\phi_1(n)}^{(1)} \right)$ qui tend vers $\lambda_1 \in K_1$.

De même façon pour tout $p \in \mathbb{N}$, on extrait une sous-suite $\left( x_{\phi_0 \circ \ldots \circ \phi_p(n)}^{(p)} \right)$ qui tend vers $\lambda_p \in K_p$.

On a donc une sous-suite de $(x_n)_{n \in \mathbb{N} }\in K^\mathbb{N}$, en sachant que toutes les sous-suites d'une suite convergente convergent vers la même valeur : 
\[
    x_{\phi_0 \circ \ldots \circ \phi_p(n)} = \left( x_{\phi_0 \circ \ldots \circ \phi_p(n)}^{(0)} , \ldots, x_{\phi_0 \circ \ldots \circ \phi_p(n)}^{(p)}\right) \underset{n \rightarrow +\infty} {\buildrel d \over \longrightarrow}\left( \lambda_0 , \ldots, \lambda_p\right) \in \prod_{t=0}^p K_t = K
\]

On a trouvé une sous-suite qui converge vers un limite dans $K$. Donc, pour n'importe quelle distance $d$, $K$ est toujours un ensemble compact.

\noindent \textit{Résolution avec la distance définie}

Avec $d$ définie dans l'énoncé, on veut montrer que, en notant $\psi = \phi_0 \circ \ldots \phi_p$ et $\Lambda = (\lambda_0, \ldots, \lambda_p)$ :
\[
    d(x_{\psi(n)}, \Lambda) = \sum_{p=0}^{\infty} \frac{1}{2^p} . \min(1, d_p(x_{\psi(n)}^{(p)}, \lambda_p))\underset{n \rightarrow +\infty} {\buildrel  \over \longrightarrow} 0
\]

Dans la partie \textit{Résolution générale}, on a déjà parlé de la convergence de $(x_{\psi(n)}^{(p)})_{n \in \mathbb{N}}$ vers $\lambda_p$ pour n'importe quel valeur de  $p$.

En conséquence, pour n'importe quell valeur de $\varepsilon_0$, on peut toujours en déduire les valeurs (dépendant de $\varepsilon_0$) $(N_1, \ldots, N_p)$, lorsque $\psi(n) > N = \max(N_1, \ldots, N_p)$, ici $N$ est un nombre entier fini car tous les  $N_1, \ldots, N_p$ sont finis :
\[
    d_p(x_{\psi(n)}^{(p)}, \lambda_p) \leq \varepsilon_0
\]

Soit $\varepsilon >0$, on prend $\varepsilon_0 = \frac{1}{2} \varepsilon$ et ensuite on obtient le valeur de  $N$ (la méthode est dans la paragraphe au-dessus), lorsque $n> N$, 
\[
d(x_{\psi(n)}, \Lambda) \leq \sum_{p=0}^{\infty} \frac{1}{2^p} . \varepsilon_0 = \sum_{p=0}^{\infty} \frac{1}{2^p} \frac{1}{2} \varepsilon < \varepsilon
\]

Donc, $d(x_{\psi(n)}, \Lambda) \underset{n \rightarrow +\infty} {\buildrel  \over \longrightarrow} 0$.

% ------------------------------------------------------------------------------

\end{document}
